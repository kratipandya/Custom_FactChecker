\documentclass{article}
\usepackage[utf8]{inputenc}
\title{Claims for basicsEvs}
\author{Auto-generated}
\date{}

\begin{document}
\maketitle

\section*{Claims}
\begin{enumerate}
\item Basics of Environmental Science introduces environmental study. \cite[p.~2]{basicsEvs}.
\item The book provides an essential understanding of natural environments and their functions. \cite[p.~2]{basicsEvs}.
\item It covers all aspects of environmental sciences with concise explanations. \cite[p.~2]{basicsEvs}.
\item Major environmental issues such as global warming are explained in detail. \cite[p.~2]{basicsEvs}.
\item Descriptions of ten major biomes are included in the book. \cite[p.~2]{basicsEvs}.
\item Michael Allaby authored or co-authored over 60 books on environmental science. \cite[p.~2]{basicsEvs}.
\item Allaby edited or co-edited seven scientific dictionaries and an anthology. \cite[p.~2]{basicsEvs}.
\item The second edition was published in 2000 by Routledge. \cite[p.~5]{basicsEvs}.
\item The book is available in both hardcover and paperback formats. \cite[p.~1]{basicsEvs}.
\item It includes a list of figures and tables for reference. \cite[p.~326]{basicsEvs}.
\item Preface and usage guide are provided at the beginning of the book. \cite[p.~69]{basicsEvs}.
\item Structure of the Earth is depicted in Figure 2.1. \cite[p.~37]{basicsEvs}.
\item Plate structure of the Earth and seismically active zones are shown in Figure 2.2. \cite[p.~39]{basicsEvs}.
\item The mountain-forming events in Europe are illustrated in Figure 2.3. \cite[p.~84]{basicsEvs}.
\item Stages in the development of an unconformity are presented in Figure 2.4. \cite[p.~157]{basicsEvs}.
\item Gradation of clay and sand to laterite is shown in Figure 2.5. \cite[p.~222]{basicsEvs}.
\item Slope development is explained in Figure 2.6. \cite[p.~1]{basicsEvs}.
\item Deposition of sand and formation of an estuarine sand bar are detailed in Figure 2.7. \cite[p.~52]{basicsEvs}.
\item The development of a sea cliff, wave-cut platform, and wave-built terrace are described in Figure 2.8. \cite[p.~1]{basicsEvs}.
\item Average amount of solar radiation reaching the ground surface is illustrated in Figure 2.9. \cite[p.~1]{basicsEvs}.
\item The greenhouse effect is explained in Figure 2.10. \cite[p.~55]{basicsEvs}.
\item Seasons and the Earth's orbit are discussed in Figure 2.11. \cite[p.~93]{basicsEvs}.
\item World distribution of soil orders is covered in chapter 3. \cite[p.~135]{basicsEvs}.
\item Two types of terracing are described for reducing runoff. \cite[p.~9]{basicsEvs}.
\item A windbreak can reduce wind speed. \cite[p.~139]{basicsEvs}.
\item There are different types of coal mines. \cite[p.~9]{basicsEvs}.
\item Structural oil and gas traps are explained. \cite[p.~142]{basicsEvs}.
\item Blast furnaces and steel converters are discussed. \cite[p.~298]{basicsEvs}.
\item The nitrogen cycle is analyzed in chapter 4. \cite[p.~114]{basicsEvs}.
\item The carbon cycle is examined in chapter 4. \cite[p.~166]{basicsEvs}.
\item Photosynthesis is covered in chapter 4. \cite[p.~7]{basicsEvs}.
\item A simplified food web in a pond is presented. \cite[p.~176]{basicsEvs}.
\item A simplified heathland food web is introduced. \cite[p.~176]{basicsEvs}.
\item The pyramid of numbers per 1000 m² of temperate grassland is shown. \cite[p.~178]{basicsEvs}.
\item World population is estimated to reach 5.24 billion by 2025. \cite[p.~1]{basicsEvs}.
\item Genetic modification of foods has become a subject of debate. \cite[p.~271]{basicsEvs}.
\item The book has been revised for a second edition. \cite[p.~35]{basicsEvs}.
\item More information about environmental issues is now available. \cite[p.~14]{basicsEvs}.
\item New controversies have emerged regarding genetically modified foods. \cite[p.~270]{basicsEvs}.
\item Information about other environmental issues has improved over time. \cite[p.~14]{basicsEvs}.
\item The author has updated the text where necessary. \cite[p.~14]{basicsEvs}.
\item Links to web sites are included throughout the book. \cite[p.~14]{basicsEvs}.
\item Albedos vary depending on the surface type. \cite[p.~157]{basicsEvs}.
\item The incident angle affects water's albedo. \cite[p.~60]{basicsEvs}.
\item Sea water contains various ions. \cite[p.~118]{basicsEvs}.
\item Oak forests consist of different minerals. \cite[p.~318]{basicsEvs}.
\item Blackbirds' diets include specific items. \cite[p.~64]{basicsEvs}.
\item The book introduces most environmental science topics in six chapters. \cite[p.~315]{basicsEvs}.
\item Each chapter contains 62 sections, numbered sequentially. \cite[p.~16]{basicsEvs}.
\item The greenhouse effect is mentioned in section 11 but requires further explanation. \cite[p.~16]{basicsEvs}.
\item A glossary defines unfamiliar terms, including those found in the index. \cite[p.~16]{basicsEvs}.
\item The index helps locate definitions for terms not covered in specific sections. \cite[p.~16]{basicsEvs}.
\item The book's structure allows readers to skip sections if they are not interested. \cite[p.~230]{basicsEvs}.
\item The introduction provides an overview of environmental science disciplines and related concepts. \cite[p.~315]{basicsEvs}.
\item The author hopes the revised edition will be valuable and interesting to readers. \cite[p.~291]{basicsEvs}.
\item The book aims to broaden understanding of environmental science science. \cite[p.~14]{basicsEvs}.
\item Technical terms may require reference to the index or glossary for full explanation. \cite[p.~16]{basicsEvs}.
\item The preface offers guidance on how to use the book effectively. \cite[p.~28]{basicsEvs}.
\item The book covers physical resources, biosphere, biological resources, and environmental management. \cite[p.~16]{basicsEvs}.
\item Environmental science has evolved over time. \cite[p.~27]{basicsEvs}.
\item In the past, educated individuals were expected to discuss various topics confidently. \cite[p.~18]{basicsEvs}.
\item The term 'science' originally meant knowledge in Latin and 'Wissenschaft' in German. \cite[p.~18]{basicsEvs}.
\item Specialization in science has increased due to accumulating discoveries. \cite[p.~315]{basicsEvs}.
\item Scientists now often struggle to understand other fields within their own specialization. \cite[p.~29]{basicsEvs}.
\item There is a need for interdisciplinary collaboration among different scientific fields. \cite[p.~298]{basicsEvs}.
\item The information explosion is largely driven by scientific research. \cite[p.~159]{basicsEvs}.
\item Educated people today generally lack detailed knowledge in specific scientific areas. \cite[p.~290]{basicsEvs}.
\item Specialists tend to know more about less in their narrow fields. \cite[p.~199]{basicsEvs}.
\item General understanding of broad scientific topics is needed. \cite[p.~283]{basicsEvs}.
\item Communication barriers exist between different scientific disciplines. \cite[p.~29]{basicsEvs}.
\item Earth's cycles involve movement from place to place. \cite[p.~232]{basicsEvs}.
\item Rock beneath feet moves due to erosion, subduction, and volcanic activity. \cite[p.~21]{basicsEvs}.
\item Cycling rates vary significantly across different parts of the cycle. \cite[p.~21]{basicsEvs}.
\item Dust particles in the troposphere have a residence time of a few weeks. \cite[p.~21]{basicsEvs}.
\item Water molecules in the air have a residence time of around 9 or 10 days. \cite[p.~21]{basicsEvs}.
\item Stratospheric water can remain for several years. \cite[p.~21]{basicsEvs}.
\item Groundwater can remain for up to 400 years. \cite[p.~21]{basicsEvs}.
\item Deep ocean water takes 1000 to 1600 years to return to the surface in the Pacific Ocean. \cite[p.~21]{basicsEvs}.
\item Radioisotopes are used for labelling in environmental monitoring. \cite[p.~295]{basicsEvs}.
\item Chemically inert dyes are used in water samples for analysis. \cite[p.~21]{basicsEvs}.
\item Radioisotopes have the same chemical properties as their non-radioactive counterparts. \cite[p.~286]{basicsEvs}.
\item Particles can be tracked using chemical or radioactive labels. \cite[p.~323]{basicsEvs}.
\item Smoke from factories can reach the ground quickly on rainy days. \cite[p.~21]{basicsEvs}.
\item Aircraft emissions take longer to reach the ground due to altitude and air conditions. \cite[p.~21]{basicsEvs}.
\item Most pollutants have short atmospheric residence times, often measured in days or weeks. \cite[p.~21]{basicsEvs}.
\item Sulphur dioxide typically remains in the air for less than a month after release. \cite[p.~21]{basicsEvs}.
\item Water's atmospheric residence time depends on evaporation rates and precipitation patterns. \cite[p.~21]{basicsEvs}.
\item Carbon-14 is formed in the atmosphere through cosmic radiation and decays over time. \cite[p.~22]{basicsEvs}.
\item The ratio of 12C to 14C in water changes when exposed to air and isolated from it. \cite[p.~22]{basicsEvs}.
\item Age of water can be estimated from its 14C content assuming certain conditions are met. \cite[p.~22]{basicsEvs}.
\item Biogeochemical cycles involve various elements essential for life. \cite[p.~164]{basicsEvs}.
\item Global systems involve interconnected components forming a coherent whole. \cite[p.~22]{basicsEvs}.
\item Your body functions as a system with various organs working together. \cite[p.~22]{basicsEvs}.
\item The Earth's surface is divided into four distinct regions: lithosphere, hydrosphere, atmosphere, and biosphere. \cite[p.~22]{basicsEvs}.
\item Materials cycle through these Earth spheres, originating in the lithosphere and ending up in the oceans. \cite[p.~24]{basicsEvs}.
\item Biogeochemical cycles are integral parts of an overarching global system. \cite[p.~22]{basicsEvs}.
\item Volcanic eruptions and plate tectonics contribute to the cycling of materials within the Earth's systems. \cite[p.~97]{basicsEvs}.
\item Organisms adapt to changing conditions by modifying their needs and strategies for survival. \cite[p.~22]{basicsEvs}.
\item The global system is not solely driven by physical forces, as organisms play a role in material cycling. \cite[p.~22]{basicsEvs}.
\item Limestone and chalk rocks form due to chemical reactions involving carbon dioxide dissolved in rainwater. \cite[p.~23]{basicsEvs}.
\item Weathering and erosion are essential processes in the cycling of materials within the Earth's systems. \cite[p.~323]{basicsEvs}.
\item The Earth's systems are complex and interdependent, raising questions about their driving forces. \cite[p.~213]{basicsEvs}.
\item The Sun has grown hotter by 25 to 30 percent since the Earth formed. \cite[p.~23]{basicsEvs}.
\item Carbon dioxide removal helps prevent surface temperatures from rising too high. \cite[p.~23]{basicsEvs}.
\item James Lovelock proposed the Gaia hypothesis. \cite[p.~208]{basicsEvs}.
\item Lovelock concluded that Earth's biogeochemical cycles are driven by biology. \cite[p.~23]{basicsEvs}.
\item Lovelock reasoned that life modifies its environment through taking materials and discharging wastes. \cite[p.~23]{basicsEvs}.
\item Earth's atmosphere is far from chemical equilibrium due to unusual amounts of nitrogen, oxygen, and methane. \cite[p.~24]{basicsEvs}.
\item Lovelock believed that organisms produce and maintain conditions favorable to themselves. \cite[p.~24]{basicsEvs}.
\item The Gaia hypothesis suggests the Earth is a single living organism. \cite[p.~24]{basicsEvs}.
\item This hypothesis has generated significant interest but remains controversial. \cite[p.~24]{basicsEvs}.
\item Lovelock named his hypothesis after the Greek goddess of the Earth. \cite[p.~29]{basicsEvs}.
\item The Gaia hypothesis posits that the Earth manages itself to maintain conditions suitable for life. \cite[p.~24]{basicsEvs}.
\item Biological processes drive biogeochemical cycles. \cite[p.~168]{basicsEvs}.
\item The Gaia hypothesis explains hospitable environments. \cite[p.~214]{basicsEvs}.
\item Iron enrichment stimulates marine plankton growth. \cite[p.~24]{basicsEvs}.
\item Biota significantly shapes its environment. \cite[p.~68]{basicsEvs}.
\item Grasslands are maintained by grazing herbivores. \cite[p.~24]{basicsEvs}.
\item Gaseous oxygen results from photosynthesis. \cite[p.~24]{basicsEvs}.
\item Humans modify their surroundings through daily activities. \cite[p.~151]{basicsEvs}.
\item Environmental changes favor certain species. \cite[p.~278]{basicsEvs}.
\item All life alters its surroundings. \cite[p.~224]{basicsEvs}.
\item Ecology and environmentalism have emerged due to environmental concerns. \cite[p.~33]{basicsEvs}.
\item We can measure concentrations of substances in air, water, soil, or food. \cite[p.~25]{basicsEvs}.
\item Pollutants are substances that are harmful and introduced by human activities. \cite[p.~25]{basicsEvs}.
\item Determining the seriousness of pollution problems requires considering costs and benefits. \cite[p.~300]{basicsEvs}.
\item Threshold doses are difficult to calculate for pollutants and their effects on humans. \cite[p.~286]{basicsEvs}.
\item Epidemiology studies help identify health impacts of pollutants on human populations. \cite[p.~326]{basicsEvs}.
\item The Chernobyl accident might cause a negligible increase in cancer deaths worldwide. \cite[p.~147]{basicsEvs}.
\item EU sets very low limits for pesticide residues to ensure safety. \cite[p.~25]{basicsEvs}.
\item Statistical risk assessment often lacks precision but still guides precautionary measures. \cite[p.~67]{basicsEvs}.
\item Caesar banned wheeled traffic from Rome's center during daytime. \cite[p.~30]{basicsEvs}.
\item Claudius expanded the ban to all Italian towns. \cite[p.~89]{basicsEvs}.
\item Marcus Aurelius applied the ban to all towns in the Roman Empire. \cite[p.~30]{basicsEvs}.
\item Hadrian further restricted vehicle entry into Rome at night. \cite[p.~53]{basicsEvs}.
\item Population density led to traffic congestion in ancient Rome. \cite[p.~30]{basicsEvs}.
\item Urban air pollution was recognized as a problem in medieval London. \cite[p.~215]{basicsEvs}.
\item Edward I enacted laws to curb smoke emissions in 1273. \cite[p.~30]{basicsEvs}.
\item Richard II initiated early efforts to reduce Thames pollution. \cite[p.~31]{basicsEvs}.
\item Elizabeth I refused to enter London due to smoke pollution. \cite[p.~31]{basicsEvs}.
\item Law originally referred to land outside enclosed farmlands for hunting. \cite[p.~31]{basicsEvs}.
\item Much of the 'forest' belonged to the sovereign and had special laws. \cite[p.~31]{basicsEvs}.
\item Forests were considered dark and dangerous places inhabited by wild animals and bandits. \cite[p.~31]{basicsEvs}.
\item Elizabethan writers used 'wilderness' to describe unmanaged forest areas. \cite[p.~31]{basicsEvs}.
\item Until modern times, famine was a risk due to weediness and crop health. \cite[p.~31]{basicsEvs}.
\item Mountains, upland moors, and wetlands were considered uncultivable wastelands. \cite[p.~31]{basicsEvs}.
\item Arthur Young advocated for the cultivation of 'waste' land in England. \cite[p.~31]{basicsEvs}.
\item Colonial expansion led to early conservation efforts in tropical regions. \cite[p.~31]{basicsEvs}.
\item Deforestation was linked to local climate change by French reformers. \cite[p.~32]{basicsEvs}.
\item Scientists in British territories observed similar relationships between deforestation and climate. \cite[p.~32]{basicsEvs}.
\item Forest reserves were established in Tobago and St Vincent in the late 18th century. \cite[p.~1]{basicsEvs}.
\item A law protecting forests in French Mauritius was enacted in 1769. \cite[p.~32]{basicsEvs}.
\item Earth is the only planet known to support life among the nine planets in the solar system. \cite[p.~36]{basicsEvs}.
\item All materials used by humans originate from the Earth. \cite[p.~36]{basicsEvs}.
\item The Earth provides food and water for all living beings. \cite[p.~104]{basicsEvs}.
\item Solar energy powers Earth's climates and biological systems. \cite[p.~36]{basicsEvs}.
\item The Earth receives energy from the Sun, driving its climates and biological processes. \cite[p.~36]{basicsEvs}.
\item The Earth supports life, making it unique among the planets in the solar system. \cite[p.~36]{basicsEvs}.
\item The Earth's surface is composed of various materials including rocks and minerals. \cite[p.~38]{basicsEvs}.
\item Weathering is a process that affects the Earth's surface over time. \cite[p.~137]{basicsEvs}.
\item Landforms on the Earth evolve through geological processes over long periods. \cite[p.~49]{basicsEvs}.
\item Coasts, estuaries, and changes in sea levels are features of the Earth's coastal areas. \cite[p.~51]{basicsEvs}.
\item The Earth's atmosphere contains oxygen, nitrogen, and other gases essential for life. \cite[p.~68]{basicsEvs}.
\item The Earth's atmosphere influences weather patterns and climate conditions. \cite[p.~81]{basicsEvs}.
\item Ice ages and interglacials occur periodically due to Earth's climate changes. \cite[p.~90]{basicsEvs}.
\item Human activities can lead to climate change affecting Earth's ecosystems. \cite[p.~204]{basicsEvs}.
\item Earth is our environment at the most fundamental level. \cite[p.~36]{basicsEvs}.
\item The oldest rocks on the Moon are about 4.6 billion years old. \cite[p.~36]{basicsEvs}.
\item René Descartes proposed the theory that the solar system formed from a cloud of gas and dust. \cite[p.~36]{basicsEvs}.
\item Most meteorites entering the Earth's atmosphere are vaporized due to friction. \cite[p.~36]{basicsEvs}.
\item The Earth-Moon system resulted from a collision between the proto-Earth and a very large body. \cite[p.~37]{basicsEvs}.
\item Lunar rocks 4.6 billion years old indicate the age of the Earth and Moon. \cite[p.~37]{basicsEvs}.
\item The primitive solar nebula may have been perturbed by material from a supernova explosion. \cite[p.~36]{basicsEvs}.
\item Fusion processes within stars convert hydrogen to helium and produce heavier elements up to iron. \cite[p.~36]{basicsEvs}.
\item The inner planets formed by accretion, with small particles moving closer and growing into larger bodies. \cite[p.~130]{basicsEvs}.
\item The Earth and Moon are considered to be of the same age due to the collision that created the Earth-Moon system. \cite[p.~37]{basicsEvs}.
\item The material of Earth became arranged in discrete layers, similar to the skins of an onion. \cite[p.~37]{basicsEvs}.
\item Earth's mean density is 5.517 g/cm³. \cite[p.~37]{basicsEvs}.
\item The Earth's inner core is composed of iron with some nickel. \cite[p.~37]{basicsEvs}.
\item The Earth's outer core is liquid iron with nickel. \cite[p.~37]{basicsEvs}.
\item The Earth's mantle is dense but somewhat plastic rock. \cite[p.~37]{basicsEvs}.
\item The Earth's crust is a thin layer of solid rock. \cite[p.~37]{basicsEvs}.
\item The geothermal gradient increases with depth. \cite[p.~37]{basicsEvs}.
\item The Earth's magnetic field is generated by movement in the outer core. \cite[p.~37]{basicsEvs}.
\item The Earth's surface area is approximately 149 million km². \cite[p.~37]{basicsEvs}.
\item Land makes up 29.22% of the Earth's surface. \cite[p.~37]{basicsEvs}.
\item Antarctica is a large continent. \cite[p.~159]{basicsEvs}.
\item The Earth's polar circumference is shorter than its equatorial circumference. \cite[p.~37]{basicsEvs}.
\item Miners found the temperature increased with depth in their underground works. \cite[p.~37]{basicsEvs}.
\item Heat increases by 20 to 40 degrees Celsius for every kilometer of depth. \cite[p.~37]{basicsEvs}.
\item In some areas, the temperature increase is only 9 or 10 degrees Celsius per kilometer. \cite[p.~215]{basicsEvs}.
\item Geothermal energy can be harnessed in volcanic regions. \cite[p.~38]{basicsEvs}.
\item Drilling techniques can bring hot water to the surface for energy purposes. \cite[p.~38]{basicsEvs}.
\item Dry subsurface rock can be hotter than its surroundings. \cite[p.~38]{basicsEvs}.
\item Exploitation of geothermal energy can be expensive due to costs associated with drilling and containment. \cite[p.~38]{basicsEvs}.
\item Substances from the rock can dissolve into the hot water. \cite[p.~38]{basicsEvs}.
\item The solution to corrosion involves keeping the water isolated from the environment. \cite[p.~38]{basicsEvs}.
\item The energy from geothermal sources is not renewable. \cite[p.~298]{basicsEvs}.
\item Over long periods, the crustal material is constantly being rearranged by mantle convection. \cite[p.~38]{basicsEvs}.
\item Subsurface heat has an indirect effect on the climate over very long time scales. \cite[p.~38]{basicsEvs}.
\item Earthquakes cause damage to physical structures leading to injuries. \cite[p.~40]{basicsEvs}.
\item Underwater earthquakes generate tsunamis affecting the entire water column. \cite[p.~40]{basicsEvs}.
\item Tsunamis rise significantly upon entering shallow waters, causing destruction. \cite[p.~249]{basicsEvs}.
\item Volcanic ash can lead to climatic cooling if it reaches the stratosphere. \cite[p.~40]{basicsEvs}.
\item Volcanic eruptions can benefit agriculture by enriching soil with minerals. \cite[p.~40]{basicsEvs}.
\item Farmers can cultivate fields near active volcanoes due to volcanic ash's fertility. \cite[p.~40]{basicsEvs}.
\item Surtsey, a new submarine volcano, erupted violently in 1963, forming an island. \cite[p.~40]{basicsEvs}.
\item The lava cone of Surtsey rose above the sea surface after cooling. \cite[p.~40]{basicsEvs}.
\item Sea birds settled on Surtsey, initiating colonization by plants and animals. \cite[p.~41]{basicsEvs}.
\item After the 1883 eruption of Krakatau, some plant species managed to survive and thrive. \cite[p.~41]{basicsEvs}.
\item By 1906, Krakatau had developed a thick forest ecosystem. \cite[p.~41]{basicsEvs}.
\item Some lizards lived on Krakatau in 1908. \cite[p.~41]{basicsEvs}.
\item Twenty-two species of animals lived on Krakatau in 1908. \cite[p.~41]{basicsEvs}.
\item Two hundred and nine species of animals lived on Krakatau in 1933. \cite[p.~41]{basicsEvs}.
\item Rats were introduced to Krakatau in 1918. \cite[p.~41]{basicsEvs}.
\item Rock salt can form salt domes beneath denser rocks. \cite[p.~41]{basicsEvs}.
\item Salt domes can sometimes break through the Earth's surface. \cite[p.~41]{basicsEvs}.
\item Salt domes can flow downhill like glaciers when they break through the surface. \cite[p.~41]{basicsEvs}.
\item The chemical composition of a rock determines its color. \cite[p.~41]{basicsEvs}.
\item Melanocratic rocks are rich in iron and magnesium compounds. \cite[p.~41]{basicsEvs}.
\item Leucocratic rocks are rich in silica compounds. \cite[p.~41]{basicsEvs}.
\item Crystals form as atoms bond to specific sites on a seed crystal's surface. \cite[p.~41]{basicsEvs}.
\item The slower a molten rock cools, the larger the crystals it is likely to contain. \cite[p.~41]{basicsEvs}.
\item Basalt forms most of the ocean floor. \cite[p.~80]{basicsEvs}.
\item Granite is a common intrusive igneous rock. \cite[p.~41]{basicsEvs}.
\item Mountain chains can form due to tectonic plate movements. \cite[p.~41]{basicsEvs}.
\item The Himalayas were formed by the collision of the Indian and Eurasian plates. \cite[p.~41]{basicsEvs}.
\item The British landscape has been shaped by multiple orogenies over millions of years. \cite[p.~42]{basicsEvs}.
\item The Caledonian-Appalachian mountain chain resulted from the first Scottish orogeny. \cite[p.~42]{basicsEvs}.
\item The Acadian orogeny occurred approximately 360 million years ago. \cite[p.~42]{basicsEvs}.
\item The Hercynian and Uralian orogenies happened around the same time as the Alleghanian. \cite[p.~42]{basicsEvs}.
\item Bosses and batholiths are exposed igneous intrusions. \cite[p.~41]{basicsEvs}.
\item Fossil shells indicate that mountains can form from folded pre-existing rocks. \cite[p.~42]{basicsEvs}.
\item Dartmoor and Bodmin Moor are located on granite batholiths. \cite[p.~42]{basicsEvs}.
\item Temperatures also undergo weathering. \cite[p.~44]{basicsEvs}.
\item Weathering affects rocks through various processes including physical and chemical changes. \cite[p.~323]{basicsEvs}.
\item Weathering can occur both above and below ground surfaces. \cite[p.~44]{basicsEvs}.
\item Oxygen and carbon dioxide in air contribute to weathering. \cite[p.~68]{basicsEvs}.
\item Water plays a crucial role in weathering through dissolution and chemical reactions. \cite[p.~166]{basicsEvs}.
\item Chemical weathering involves dissolution, oxidation, hydration, and hydrolysis processes. \cite[p.~44]{basicsEvs}.
\item Limestone pavements are visible evidence of chemical weathering. \cite[p.~44]{basicsEvs}.
\item Red sandstones in Devon were originally arid desert sands rich in iron. \cite[p.~45]{basicsEvs}.
\item Iron in these sands was oxidized to form the red color of the sandstone. \cite[p.~45]{basicsEvs}.
\item Limestone pavements typically form in areas where limestone beds are exposed and eroded. \cite[p.~201]{basicsEvs}.
\item Weathering leads to the formation of distinctive geological features such as limestone pavements. \cite[p.~44]{basicsEvs}.
\item ic acid reacts with calcium carbonate to produce calcium bicarbonate. \cite[p.~45]{basicsEvs}.
\item Calcium bicarbonate is soluble in water and forms deep crevices. \cite[p.~45]{basicsEvs}.
\item Grikes provide a habitat for lime-loving plants. \cite[p.~45]{basicsEvs}.
\item Limestone pavements are protected by law to prevent their destruction. \cite[p.~45]{basicsEvs}.
\item Iron oxidizes to form hematite, an important iron ore mineral. \cite[p.~45]{basicsEvs}.
\item Banded ironstone formations contain alternating bands of hematite and chert. \cite[p.~45]{basicsEvs}.
\item Nodules of iron, manganese, and other metals form near mid-ocean ridges. \cite[p.~45]{basicsEvs}.
\item Vast fields of iron and metal nodules exist on ocean floors. \cite[p.~45]{basicsEvs}.
\item Serious consideration was once given to harvesting these nodules from the ocean floor. \cite[p.~139]{basicsEvs}.
\item Kaolin is used in the production of fine porcelain and as a filler in paper. \cite[p.~147]{basicsEvs}.
\item Today, most kaolin is used as a filler and whitener in paper. \cite[p.~45]{basicsEvs}.
\item Most extensively mined kaolin deposits are located in Cornwall and Devon, Britain. \cite[p.~45]{basicsEvs}.
\item Kaolin is a hydrated aluminium silicate, Al₂O₃·2SiO₂·2H₂O, derived from the mineral kaolinite. \cite[p.~45]{basicsEvs}.
\item British kaolinite deposits are associated with granite batholiths and bosses intruded during the Hercynian orogeny. \cite[p.~45]{basicsEvs}.
\item Granites contain quartz, mica, and feldspars, with feldspars being variable in composition and often rich in sodium. \cite[p.~45]{basicsEvs}.
\item Kaolinite forms as minute white hexagonal plates that are separated from the rock by industrial washing and precipitation, leaving behind quartz grains and mica. \cite[p.~46]{basicsEvs}.
\item Approximately 15% of the material recovered is kaolin, 10% is mica waste, and 75% is sand, which has found some use in building and landscaping. \cite[p.~46]{basicsEvs}.
\item Bauxite, an important ore of aluminium, is produced by the chemical weathering of feldspars, typically containing 25-30% aluminium oxide. \cite[p.~46]{basicsEvs}.
\item Bauxite is a mixture of hydrous aluminium oxides and hydroxides with various metals as impurities, suitable for mining if it contains 25-30% aluminium oxide. \cite[p.~46]{basicsEvs}.
\item Laterization, a type of extreme weathering of soil, occurs primarily in some parts of the seasonal tropics where soils are derived from granite parent material. \cite[p.~46]{basicsEvs}.
\item Laterization can be triggered by removing the forest or other natural vegetation in certain areas, potentially leading to the formation of laterites. \cite[p.~46]{basicsEvs}.
\item Laterites are brick-hard and can be broken by ploughing, except on steep slopes. \cite[p.~46]{basicsEvs}.
\item ils overlying granite can be up to 30 m deep. \cite[p.~46]{basicsEvs}.
\item naturally acidic water from the surface percolates through them. \cite[p.~46]{basicsEvs}.
\item plants draw the water up again through their roots. \cite[p.~46]{basicsEvs}.
\item water is also drawn upward by capillary attraction through tiny spaces between soil particles. \cite[p.~46]{basicsEvs}.
\item if the rainfall is fairly constant through the year, the movement of water is also constant. \cite[p.~46]{basicsEvs}.
\item if it is strongly seasonal, evaporation exceeds precipitation during the dry season. \cite[p.~46]{basicsEvs}.
\item mineral compounds dissolved in the soil water are precipitated, the least soluble being precipitated first. \cite[p.~46]{basicsEvs}.
\item provided vegetation cover is adequate, with roots penetrating deep into the soil, the minerals will not accumulate in particular places. \cite[p.~46]{basicsEvs}.
\item if there is little plant cover, however, they may accumulate near the surface. \cite[p.~46]{basicsEvs}.
\item the most insoluble minerals are hydroxides of iron and aluminium (kaolinite). \cite[p.~46]{basicsEvs}.
\item they are what give many tropical soils their typically red or yellow color. \cite[p.~46]{basicsEvs}.
\item Soil formation involves biological activity. \cite[p.~47]{basicsEvs}.
\item Physical weathering plays a role in soil formation. \cite[p.~47]{basicsEvs}.
\item Thermal weathering leads to rock fragmentation and particle detachment. \cite[p.~44]{basicsEvs}.
\item Water erosion is the most significant form of soil erosion. \cite[p.~140]{basicsEvs}.
\item Human activities contribute to soil degradation. \cite[p.~164]{basicsEvs}.
\item The UN estimates global land degradation by water erosion at 1.093 billion hectares. \cite[p.~47]{basicsEvs}.
\item Sheet and surface erosion affect 920 million hectares globally. \cite[p.~47]{basicsEvs}.
\item Rills and gullies degrade an additional 173 million hectares. \cite[p.~47]{basicsEvs}.
\item Natural vegetation removal and deforestation cause 43% of water erosion. \cite[p.~47]{basicsEvs}.
\item Over-grazing and poor farming practices contribute to 29% of water erosion. \cite[p.~47]{basicsEvs}.
\item Modern farming techniques can significantly reduce soil erosion. \cite[p.~47]{basicsEvs}.
\item Rock is recycled through various natural processes to form soils and landscapes. \cite[p.~47]{basicsEvs}.
\item Human activities can speed up the natural process of landscape alteration on vulnerable land. \cite[p.~47]{basicsEvs}.
\item Landscapes evolve due to weathering of rocks and erosion of loose particles. \cite[p.~47]{basicsEvs}.
\item The 1952 Lynmouth flood was an example of sudden landscape change. \cite[p.~47]{basicsEvs}.
\item During the last glaciation, Dartmoor experienced severe cold with permanent frozen ground. \cite[p.~47]{basicsEvs}.
\item Permafrost caused rock masses on Dartmoor to break apart due to repeated freezing and thawing of water. \cite[p.~47]{basicsEvs}.
\item The permafrost thawed for a few weeks in summer, causing mud and boulders to slide downhill. \cite[p.~48]{basicsEvs}.
\item Today, remnants of permafrost on Dartmoor provide evidence of past climates. \cite[p.~98]{basicsEvs}.
\item Similar periglacial processes affected the weak, jointed chalk in southern England. \cite[p.~48]{basicsEvs}.
\item These processes resulted in the formation of deposits called 'coombe rock. \cite[p.~41]{basicsEvs}.
\item Periglacial phenomena have been observed in North America and other European regions. \cite[p.~48]{basicsEvs}.
\item The 1952 Lynmouth flood occurred after heavy rainfall in Exmoor, England. \cite[p.~47]{basicsEvs}.
\item of 15 and 16 August both rivers flooded. \cite[p.~48]{basicsEvs}.
\item Overflow from the West Lyn created a new channel. \cite[p.~48]{basicsEvs}.
\item Houses, roads, and bridges were destroyed during the flood. \cite[p.~48]{basicsEvs}.
\item An estimated 40,000 tons of debris were left in Lynmouth. \cite[p.~209]{basicsEvs}.
\item 31 people lost their lives in the flooding disaster. \cite[p.~147]{basicsEvs}.
\item The flooding was primarily caused by heavy rainfall. \cite[p.~160]{basicsEvs}.
\item Presently, permafrost regions exist in high latitudes. \cite[p.~48]{basicsEvs}.
\item In Canada and Alaska, permafrost can reach depths of up to 400 meters. \cite[p.~48]{basicsEvs}.
\item In Siberia, permafrost can extend up to 700 meters deep. \cite[p.~48]{basicsEvs}.
\item Permafrost covers approximately 20% of the Arctic Circle's land area. \cite[p.~48]{basicsEvs}.
\item Ice sheets significantly shape landscapes through erosion and depression. \cite[p.~48]{basicsEvs}.
\item Geomorphology studies landforms and their formation. \cite[p.~319]{basicsEvs}.
\item William Morris Davis proposed the 'Davisian cycle' of landscape evolution. \cite[p.~49]{basicsEvs}.
\item The Davisian cycle' starts with land being raised by tectonic movements. \cite[p.~50]{basicsEvs}.
\item Hill slopes become gentler as landscapes mature. \cite[p.~50]{basicsEvs}.
\item A peneplain is described as 'almost a plain. \cite[p.~50]{basicsEvs}.
\item Walther Penck argued that slopes maintain a stable angle. \cite[p.~50]{basicsEvs}.
\item Erosion wears away the face of a slope but does not make it shallower. \cite[p.~50]{basicsEvs}.
\item Slopes eroding faster at lower angles due to weathered material protection. \cite[p.~53]{basicsEvs}.
\item Understanding slope behavior is crucial for engineering and environmental purposes. \cite[p.~19]{basicsEvs}.
\item Rivers transport eroded particles from uplands to the sea. \cite[p.~50]{basicsEvs}.
\item Landslides, erosion, and flooding are risks engineers aim to mitigate. \cite[p.~50]{basicsEvs}.
\item Water transports mineral particles. \cite[p.~50]{basicsEvs}.
\item Organic matter and dissolved plant nutrients are carried by rivers. \cite[p.~50]{basicsEvs}.
\item Rivers supply water for domestic and industrial use. \cite[p.~50]{basicsEvs}.
\item Water moves from higher to lower ground through soil layers. \cite[p.~113]{basicsEvs}.
\item The water table is the upper limit of saturated soil. \cite[p.~108]{basicsEvs}.
\item A drainage system removes water from a specific area. \cite[p.~50]{basicsEvs}.
\item A catchment in Britain refers to the area drained by a drainage system. \cite[p.~50]{basicsEvs}.
\item In estuaries, large particles settle later than small ones. \cite[p.~52]{basicsEvs}.
\item Salinity variations make it difficult for most species to survive in estuarine environments. \cite[p.~90]{basicsEvs}.
\item Ion travels only at the speed of light. \cite[p.~55]{basicsEvs}.
\item Spherical particles larger than 0.1 micrometers scatter light without changing its direction, darkening the sky after rain. \cite[p.~55]{basicsEvs}.
\item All received energy is reradiated by the Earth, and some is converted back into heat by respiration. \cite[p.~57]{basicsEvs}.
\item Gasohol' was previously declining in Brazil but saw increased production in 1999. \cite[p.~251]{basicsEvs}.
\item Methanol can be produced from plant material and does not contribute to the greenhouse effect. \cite[p.~58]{basicsEvs}.
\item Turbines occupy about 2 hectares each. \cite[p.~1]{basicsEvs}.
\item Large conventional power stations produce comparable energy output. \cite[p.~59]{basicsEvs}.
\item Energy requires significantly more to increase water's temperature compared to rock. \cite[p.~61]{basicsEvs}.
\item Water's heat can move through it easily via convection and mixing. \cite[p.~61]{basicsEvs}.
\item The greenhouse effect warms the Earth by absorbing long-wave radiation from the sun. \cite[p.~323]{basicsEvs}.
\item Nitrogen and oxygen are nearly transparent to certain types of electromagnetic radiation. \cite[p.~62]{basicsEvs}.
\item Ozone absorbs about 4% of incoming solar radiation. \cite[p.~55]{basicsEvs}.
\item At about 1.5 µm, 2.0 µm, and 2.5-4.5 µm, there are 6 per cent by water droplets and dust. \cite[p.~62]{basicsEvs}.
\item Water vapor is the most important greenhouse gas overall. \cite[p.~63]{basicsEvs}.
\item The average global temperature could rise between 1.5 and 4.5 degrees Celsius. \cite[p.~1]{basicsEvs}.
\item If at mean tide, sea level has fallen. \cite[p.~65]{basicsEvs}.
\item Models require substantial computational resources. \cite[p.~107]{basicsEvs}.
\item Water vapor's effect on warming remains uncertain. \cite[p.~80]{basicsEvs}.
\item Volcanic activity later led to the formation of a new atmosphere. \cite[p.~47]{basicsEvs}.
\item Nitrogen is chemically stable but can be oxidized by lightning and other means. \cite[p.~113]{basicsEvs}.
\item Thunderstorms contribute to the replenishment of atmospheric nitrogen annually. \cite[p.~168]{basicsEvs}.
\item Biological processes help sustain the current atmospheric composition. \cite[p.~124]{basicsEvs}.
\item Human activities such as lightning and bacterial processes influence atmospheric nitrogen levels. \cite[p.~273]{basicsEvs}.
\item At 350 km, the temperature can reach over 900 °C. \cite[p.~70]{basicsEvs}.
\item The source of water vapor in the stratosphere is unclear. \cite[p.~117]{basicsEvs}.
\item Beyond the mesosphere, atmospheric composition stabilizes due to turbulence-induced mixing. \cite[p.~71]{basicsEvs}.
\item A watershed in North America is the area drained by a drainage system. \cite[p.~50]{basicsEvs}.
\item Drainage patterns vary based on climate, rock type, and erosion extent. \cite[p.~51]{basicsEvs}.
\item Dendritic patterns typically form on gently sloping land with uniform geology. \cite[p.~51]{basicsEvs}.
\item Radial patterns occur around domed hills and batholiths. \cite[p.~51]{basicsEvs}.
\item Trellis patterns develop where rivers cross alternating bands of hard and soft rocks. \cite[p.~51]{basicsEvs}.
\item Sea water loses energy when pushing against fresh water. \cite[p.~52]{basicsEvs}.
\item Sand grains are larger and heavier than silt particles. \cite[p.~52]{basicsEvs}.
\item Particles of silt range from 2 to 60 micrometres in diameter. \cite[p.~52]{basicsEvs}.
\item Sand grains range from 60 to 2000 micrometres in diameter. \cite[p.~52]{basicsEvs}.
\item Mudbanks are formed from silt and smaller clay particles. \cite[p.~52]{basicsEvs}.
\item Organic molecules contribute to the formation of clumps in estuarine waters. \cite[p.~317]{basicsEvs}.
\item Bacteria and burrowing invertebrates thrive in estuarine mud due to organic material. \cite[p.~52]{basicsEvs}.
\item Estuaries can be enriched by nutrient traps caused by current patterns. \cite[p.~52]{basicsEvs}.
\item Mangroves help extend tropical coastlines by trapping sediment. \cite[p.~53]{basicsEvs}.
\item Salt marsh vegetation raises the surface above the tide line, creating dry land. \cite[p.~53]{basicsEvs}.
\item Solar radiation is absorbed in the upper atmosphere. \cite[p.~55]{basicsEvs}.
\item Radiation with wavelengths between 0.2 and 0.4 μm is ultraviolet. \cite[p.~55]{basicsEvs}.
\item Most UV is absorbed by stratospheric oxygen and ozone. \cite[p.~55]{basicsEvs}.
\item Visible light wavelengths range from 0.4 to 0.7 μm. \cite[p.~55]{basicsEvs}.
\item The Sun radiates most intensely in the green part of the spectrum. \cite[p.~55]{basicsEvs}.
\item Water vapor absorbs energy in several narrow bands between 0.9 and 2.1 μm. \cite[p.~55]{basicsEvs}.
\item Energy is absorbed when radiant energy strikes the Earth's surface. \cite[p.~55]{basicsEvs}.
\item The Earth is not warmed evenly due to latitude and cloudiness. \cite[p.~55]{basicsEvs}.
\item Insolation is higher in tropical and subtropical deserts than at the equator. \cite[p.~55]{basicsEvs}.
\item Clouds reflect incoming sunlight in the equatorial region. \cite[p.~55]{basicsEvs}.
\item Most of the 'lost' incoming solar radiation is reflected directly back into space. \cite[p.~55]{basicsEvs}.
\item About 10 percent of the incoming radiation is absorbed or scattered by atmospheric components. \cite[p.~55]{basicsEvs}.
\item Rayleigh scattering occurs when shorter wavelengths scatter more than longer ones. \cite[p.~55]{basicsEvs}.
\item The sky appears blue when the Sun is high due to scattering of blue light. \cite[p.~55]{basicsEvs}.
\item When the sky is hazy, dust particles scatter light of all wavelengths, making the sky appear white. \cite[p.~55]{basicsEvs}.
\item During sunrise and sunset, dust particles scatter light primarily in the orange and red wavelengths. \cite[p.~55]{basicsEvs}.
\item The Earth acts as a black body, radiating energy in the long, infrared waveband. \cite[p.~57]{basicsEvs}.
\item If the Earth retained captured energy permanently, it would grow hotter, contradicting observed temperatures. \cite[p.~57]{basicsEvs}.
\item Plants capture and pass on energy to animals, which then release it back into the environment as heat. \cite[p.~57]{basicsEvs}.
\item Solar energy can be exploited for domestic and industrial use. \cite[p.~57]{basicsEvs}.
\item Biomass crops require large amounts of land and may compete with food or fiber crops. \cite[p.~58]{basicsEvs}.
\item Manufactured solar collectors absorb heat and transfer it to hot water systems. \cite[p.~58]{basicsEvs}.
\item Solar collectors are most effective in areas with high insolation levels. \cite[p.~58]{basicsEvs}.
\item Up to 3000 turbines can be installed in an array. \cite[p.~58]{basicsEvs}.
\item Wind turbines require space to avoid mutual interference. \cite[p.~58]{basicsEvs}.
\item Wind turbines are unreliable and require backup conventional capacity. \cite[p.~59]{basicsEvs}.
\item Installation sites are often visually intrusive and politically contentious. \cite[p.~59]{basicsEvs}.
\item Wind power could potentially alter local climates significantly. \cite[p.~59]{basicsEvs}.
\item Wave power technology is well developed but faces challenges. \cite[p.~59]{basicsEvs}.
\item Wave power installations require large structures and robust infrastructure. \cite[p.~59]{basicsEvs}.
\item Wave power is limited to areas with significant wave activity. \cite[p.~59]{basicsEvs}.
\item Small-scale applications of wind and wave power are suitable for remote locations. \cite[p.~59]{basicsEvs}.
\item Solar energy may become more accessible with improved efficiency and cost-sharing strategies. \cite[p.~59]{basicsEvs}.
\item They are also produced by a range of human activities, especially the burning of fuels. \cite[p.~61]{basicsEvs}.
\item It is difficult to separate natural sources from those linked directly to human activities. \cite[p.~61]{basicsEvs}.
\item Agriculture and industry account for about one-third of the particulate matter in the air. \cite[p.~61]{basicsEvs}.
\item Forest clearance and the overgrazing of marginal land in semi-arid regions lead to large injections of small particles as wind-blown soil. \cite[p.~61]{basicsEvs}.
\item Particles injected into the upper troposphere might increase the formation of cirriform cloud. \cite[p.~61]{basicsEvs}.
\item Particles injected into the stratosphere might increase planetary albedo for several years. \cite[p.~61]{basicsEvs}.
\item Such 'thermal mountains' would stimulate convection, hopefully leading to the formation of cumuliform clouds that would release rain. \cite[p.~61]{basicsEvs}.
\item Most climatologists are wary of such schemes, suspecting that in the unlikely event that they worked the unanticipated consequences might be unpleasant. \cite[p.~61]{basicsEvs}.
\item On a really hot summer day the surface temperature of sand on a beach may be high enough to make it painful to walk across it in bare feet. \cite[p.~61]{basicsEvs}.
\item Dig your feet into the sand, however, and you soon reach a cooler level. \cite[p.~61]{basicsEvs}.
\item Different materials vary in their response to radiant energy because they have different heat capacities. \cite[p.~61]{basicsEvs}.
\item Water has a much higher thermal capacity than rock. \cite[p.~61]{basicsEvs}.
\item Water cools down more slowly than rock after losing heat. \cite[p.~61]{basicsEvs}.
\item The temperature change near the surface depends on the material's conductivity and mobility. \cite[p.~61]{basicsEvs}.
\item Sand conducts heat poorly, resulting in a cool layer close to the surface. \cite[p.~61]{basicsEvs}.
\item Without the greenhouse effect, Earth's average surface temperature would be around 250 K (-23°C). \cite[p.~61]{basicsEvs}.
\item The current average surface temperature is approximately 288 K (+15°C). \cite[p.~62]{basicsEvs}.
\item The greenhouse effect causes an average temperature difference of 38°C. \cite[p.~62]{basicsEvs}.
\item Carbon dioxide absorbs about 20% of the incoming solar radiation in the infrared band. \cite[p.~62]{basicsEvs}.
\item At wavelengths greater than about 4.0 µm, several atmospheric gases absorb radiation. \cite[p.~62]{basicsEvs}.
\item The Earth emits electromagnetic radiation at 4-100 µm, with a peak of intensity around 10 µm. \cite[p.~62]{basicsEvs}.
\item More than 90 per cent of the Earth's outgoing long-wave radiation is absorbed in the atmosphere. \cite[p.~62]{basicsEvs}.
\item The remainder, about 6 per cent, escapes into space. \cite[p.~62]{basicsEvs}.
\item These 'gaps' in the absorption bands are called the 'atmospheric window. \cite[p.~62]{basicsEvs}.
\item Molecules absorb radiation and reradiate it in all directions. \cite[p.~62]{basicsEvs}.
\item Some of the reradiated radiation returns to the surface, some is absorbed by other atmospheric molecules, and some is radiated upwards. \cite[p.~62]{basicsEvs}.
\item The surface is warmed by the Sun only during daytime, but its heat is radiated away by night as well as by day. \cite[p.~62]{basicsEvs}.
\item The overall energy budget of the Earth must balance, and it does. \cite[p.~62]{basicsEvs}.
\item Greenhouse' is a colorful but misleading metaphor. \cite[p.~62]{basicsEvs}.
\item The atmospheric greenhouse effect is real and important. \cite[p.~63]{basicsEvs}.
\item Vostok cores go back to about 160,000 years ago. \cite[p.~63]{basicsEvs}.
\item The greenhouse-gas concentration fluctuates with temperature changes. \cite[p.~63]{basicsEvs}.
\item Carbon dioxide levels increased after temperatures rose during the end of the last ice age. \cite[p.~63]{basicsEvs}.
\item Carbon dioxide measurements from ice cores are unreliable due to its solubility in ice. \cite[p.~63]{basicsEvs}.
\item The concentration of carbon dioxide was lower before the Industrial Revolution. \cite[p.~63]{basicsEvs}.
\item The Industrial Revolution caused a significant increase in atmospheric carbon dioxide concentration. \cite[p.~63]{basicsEvs}.
\item The 'warm champagne' effect occurs when temperature rises and carbon dioxide bubbles out of oceans. \cite[p.~63]{basicsEvs}.
\item The 'warm beer' effect happens as aerobic bacteria release carbon dioxide due to rising temperatures. \cite[p.~167]{basicsEvs}.
\item Carbon dioxide is the most abundant greenhouse gas over which humans can control. \cite[p.~63]{basicsEvs}.
\item Methane, nitrous oxide, and tropospheric ozone are other greenhouse gases. \cite[p.~63]{basicsEvs}.
\item The concentration of methane is about 1.7 ppm, nitrous oxide is 0.31 ppm, and tropospheric ozone is 0.06 ppm. \cite[p.~63]{basicsEvs}.
\item Its concentration varies greatly from place to place and from day to day. \cite[p.~63]{basicsEvs}.
\item It is strongly affected by temperature. \cite[p.~63]{basicsEvs}.
\item The influence of this gas tends to add to those of the other gases. \cite[p.~63]{basicsEvs}.
\item Carbon dioxide is used as a reference for calculating global warming potentials. \cite[p.~325]{basicsEvs}.
\item Methane has a global warming potential of 11 compared to carbon dioxide. \cite[p.~63]{basicsEvs}.
\item Nitrous oxide has a global warming potential of 270 compared to carbon dioxide. \cite[p.~63]{basicsEvs}.
\item CFC-11 has a global warming potential of 3400 compared to carbon dioxide. \cite[p.~63]{basicsEvs}.
\item CFC-12 has a global warming potential of 7100 compared to carbon dioxide. \cite[p.~63]{basicsEvs}.
\item A doubling of carbon dioxide concentration is used to calculate future climatic warming. \cite[p.~63]{basicsEvs}.
\item Warming of the oceans is expected to cause a rise in sea level of 2-4 cm per decade. \cite[p.~65]{basicsEvs}.
\item It is uncertain whether sea levels have risen worldwide by about 25 cm over the past century. \cite[p.~65]{basicsEvs}.
\item Methane has a global warming potential of 11 times that of carbon dioxide. \cite[p.~63]{basicsEvs}.
\item Nitrous oxide has a global warming potential of 270 times that of carbon dioxide. \cite[p.~63]{basicsEvs}.
\item CFC-11 has a global warming potential of 3400 times that of carbon dioxide. \cite[p.~63]{basicsEvs}.
\item CFC-12 has a global warming potential of 7100 times that of carbon dioxide. \cite[p.~63]{basicsEvs}.
\item Future climatic warming estimates are based on a doubling of carbon dioxide concentration. \cite[p.~63]{basicsEvs}.
\item A doubling of carbon dioxide would raise the average global temperature by 1.5-4.5°C. \cite[p.~1]{basicsEvs}.
\item Warming of the oceans due to greenhouse gases causes a rise in sea level of 2-4 cm per decade. \cite[p.~65]{basicsEvs}.
\item Predictions of global warming include an assertion that sea levels have risen worldwide by about 25 cm over the past century. \cite[p.~65]{basicsEvs}.
\item The sea-level gauge was meant to indicate sea level. \cite[p.~65]{basicsEvs}.
\item John Daly rediscovered the sea-level gauge in 1999. \cite[p.~65]{basicsEvs}.
\item Daly found the gauge still visible above the water line. \cite[p.~65]{basicsEvs}.
\item The exact setting of the gauge is uncertain. \cite[p.~65]{basicsEvs}.
\item If close to high tide, sea level hasn't changed since 1841. \cite[p.~65]{basicsEvs}.
\item Mean global temperature increased by 0.37°C from 1881 to 1940. \cite[p.~65]{basicsEvs}.
\item Temperature fell from 1940 to the 1970s, then rose again. \cite[p.~65]{basicsEvs}.
\item No clear evidence of warming between 1980 and 1998. \cite[p.~65]{basicsEvs}.
\item Total warming from 1881 to 1993 was 0.54°C. \cite[p.~65]{basicsEvs}.
\item Two-thirds of the warming occurred before 1940. \cite[p.~65]{basicsEvs}.
\item Surface measurements can be affected by urban development and staff changes. \cite[p.~65]{basicsEvs}.
\item Oceans are the most important sink for carbon dioxide. \cite[p.~66]{basicsEvs}.
\item The behavior of carbon dioxide sinks is not fully understood. \cite[p.~66]{basicsEvs}.
\item Carbon dioxide levels fluctuate seasonally due to plant growth. \cite[p.~66]{basicsEvs}.
\item General circulation models simulate the climate using physical laws. \cite[p.~66]{basicsEvs}.
\item Cloud formation cannot be accurately represented in GCMs. \cite[p.~66]{basicsEvs}.
\item Ocean mixing is simplified in most GCMs. \cite[p.~67]{basicsEvs}.
\item Coupled models treat the oceans similarly to the atmosphere complexity. \cite[p.~67]{basicsEvs}.
\item Scientific understanding of atmospheric and oceanic processes is advancing rapidly. \cite[p.~67]{basicsEvs}.
\item Regional warming impacts vary significantly. \cite[p.~67]{basicsEvs}.
\item Water vapor is a greenhouse gas. \cite[p.~67]{basicsEvs}.
\item A more humid atmosphere will be cloudier. \cite[p.~67]{basicsEvs}.
\item Clouds can have both warming and cooling effects. \cite[p.~67]{basicsEvs}.
\item Most General Circulation Models predict an increase in high-level clouds. \cite[p.~66]{basicsEvs}.
\item It is uncertain how much cloud formation will occur and what types of clouds will form. \cite[p.~67]{basicsEvs}.
\item Many climatologists believe there is a real possibility of global climatic warming. \cite[p.~67]{basicsEvs}.
\item Climate belts could shift towards higher latitudes, benefiting some regions and harming others. \cite[p.~67]{basicsEvs}.
\item Warming might lead to reduced nighttime temperatures and increased soil moisture. \cite[p.~267]{basicsEvs}.
\item Environmentalists support the precautionary principle. \cite[p.~327]{basicsEvs}.
\item The Rio summit aimed to reduce greenhouse gas emissions based on the precautionary principle. \cite[p.~67]{basicsEvs}.
\item Critics argue that implementing policies based on the precautionary principle can be costly and difficult. \cite[p.~67]{basicsEvs}.
\item Earth's early atmosphere may have consisted of hydrogen and helium. \cite[p.~68]{basicsEvs}.
\item The current atmosphere is believed to have developed significantly due to biological processes. \cite[p.~68]{basicsEvs}.
\item Denitrifying bacteria play a role in maintaining atmospheric nitrogen levels. \cite[p.~168]{basicsEvs}.
\item The Earth's atmosphere continues to evolve and change over time. \cite[p.~73]{basicsEvs}.
\item It is uncertain what the exact composition of Earth's early atmosphere was. \cite[p.~73]{basicsEvs}.
\item The presence of free oxygen in Earth's early atmosphere is questionable. \cite[p.~24]{basicsEvs}.
\item Temperature decreases with height in the troposphere. \cite[p.~82]{basicsEvs}.
\item The tropopause marks the boundary between the troposphere and stratosphere. \cite[p.~69]{basicsEvs}.
\item The stratosphere has a constant temperature region above the tropopause. \cite[p.~70]{basicsEvs}.
\item The mesosphere experiences a decrease in temperature with increasing altitude. \cite[p.~320]{basicsEvs}.
\item The thermosphere is characterized by a rise in temperature with altitude. \cite[p.~320]{basicsEvs}.
\item The mesopause is located at approximately 80 km altitude. \cite[p.~70]{basicsEvs}.
\item The density of oxygen molecules is sufficient to absorb most solar ultraviolet radiation. \cite[p.~70]{basicsEvs}.
\item Ultraviolet radiation can cause separation of oxygen molecules into oxygen atoms. \cite[p.~70]{basicsEvs}.
\item Ozone formation occurs when oxygen atoms combine with oxygen molecules. \cite[p.~70]{basicsEvs}.
\item Satellites experience measurable drag due to the thermosphere. \cite[p.~70]{basicsEvs}.
\item Atoms combine with ozone in two steps releasing free chlorine. \cite[p.~70]{basicsEvs}.
\item CFCs are broken down by UV radiation, releasing free chlorine. \cite[p.~71]{basicsEvs}.
\item The vortex in the Antarctic disappears during spring, moving ozone to higher latitudes. \cite[p.~71]{basicsEvs}.
\item Arctic ozone depletion is less severe and of shorter duration. \cite[p.~71]{basicsEvs}.
\item Ultraviolet radiation may cause cataracts and skin cancer in humans. \cite[p.~71]{basicsEvs}.
\item Ultraviolet radiation can harm land plants and organisms near the ocean surface. \cite[p.~71]{basicsEvs}.
\item Ozone is a minor component of the atmosphere, mainly found in the troposphere. \cite[p.~71]{basicsEvs}.
\item Water vapor is a minor atmospheric component, comprising up to 4% in the lower atmosphere. \cite[p.~71]{basicsEvs}.
\item Water vapor enters the stratosphere either as water vapor or from the oxidation of methane. \cite[p.~71]{basicsEvs}.
\item Earth moves westward by 50.27 seconds of arc a year. \cite[p.~73]{basicsEvs}.
\item This motion causes the precession of the equinoxes. \cite[p.~93]{basicsEvs}.
\item The precession alters the dates of perihelion and aphelion. \cite[p.~73]{basicsEvs}.
\end{enumerate}
\end{document}